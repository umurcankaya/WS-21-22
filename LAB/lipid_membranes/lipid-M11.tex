\documentclass{article}

\usepackage[final]{neurips_2020}
\usepackage[utf8]{inputenc} % allow utf-8 input
\usepackage[T1]{fontenc}    % use 8-bit T1 fonts
\usepackage{hyperref}       % hyperlinks
\usepackage[pdftex]{graphicx}
\usepackage{url}            % simple URL typesetting
\usepackage{booktabs}       % professional-quality tables
\usepackage{amsfonts}       % blackboard math symbols
\usepackage{nicefrac}       % compact symbols for 1/2, etc.
\usepackage{microtype}      % microtypography

\title{Dynamic processes in lipid membranes}

\author{
Umur Can Kaya\\
5410770\\
\texttt{umurcan.kaya@gmail.com}\\
\And
Rohit Sharma\\
5442717\\
\texttt{rohis97@zedat.fu-berlin.de}\\
}

\begin{document}


\maketitle

\begin{abstract}
There exist in nature molecular and supramolecular systems whose structural dynamics it is possible to study using fluorescent molecules as external or internal probes meaning a foreign compound can be inserted or a probe can be recognised and activated within the target system. The probes provide us an opportunity to get an insight into the fast dynamics of the system internally. We study the characterisation  the  fluorophore  probe  by  identifying  the  excitation and  emission  spectra  of  the 
fluorophore and measure the fluorescence anisotropy in a lipid membrane at different temperatures using time resolved and steady state fluorescence methods.
\end{abstract}

\section{Theory}
Flourescense occurs when an excited system relaxes to a lower energy state of the same spin state through emission of a photon. Light particles are absorbed in certain molecules in a direction corresponding to the electronic state.The region of the molecule responsible for absorption are called chromosphores. Irrespective of the state of excitation of the molecule due to the movement towards electron pairs the dipole length (known as transition dipole moment) remains the same. When EM wave of linear polarisation is incident on  fluorophores group, elements of the group whose transition moment is in the corresponding direction as that of the electric vector of the incident beam are excited [1]. Since this is the case the  fluorescence is anisotropic (directionally dependent properties). The fluorescence revels information about steric restrictions which is basically the slowing of chemical reactions and intramolecular factors pertaining to it. \\

Biological membranes engineer their surroundings are differentiated from it by creating cells.They also regulate the passing of relevant substances, ions for production of energy  They consist of sheets called phospholipid bilayer made up of lipid molecules. Diphenylhexatrien (DPH) at room temperature transition from a gel like ordered state to a liquid-crystal like disordered state. We use the DPH as a probe in DiMyristolphosphatidylcholin (DMPC),a vesicle membrane meaning nanoparticles composed of lipid membranes to study the faster decrease of the relationship of electric vector orientation and corresponding excitation in the disordered state. According to the literature,fluorescence intensity of DPH are negatively affected by the polar water molecules so that its fluorescence is 
only  ~  1/200  of  DPH  fluorescence  in  a  hydrophobic  environment.  Hence,  fluorescence  of  free  DPH  in 
solution can be neglected.




\section{Tasks}
\subsection{Experimental setup}
\subsection{Sample preparation}
\subsection{Fluorescence excitation and emission spectra}
\subsection{Steady state fluorescence depolarization}
\subsection{Time resolved fluorescence depolarization of DPH in DMPC vesicles}
\section{Discussion}

%\begin{figure}[h!]
%\centering
%\includegraphics[width=0.6\linewidth]{example.pdf}
%\caption{Sample figure caption.}
%\end{figure}


\section*{References}

[1] Ma 5: Dynamic Processes in lipidmembranes, https://wiki.physik.fu-berlin.de/fp/private:ma5 \_dynamic\_processes\_in\_lipid\_membranes

\end{document}
