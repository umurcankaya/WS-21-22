%----------------------------------------------------------------------------------------
%	PACKAGES AND THEMES
%----------------------------------------------------------------------------------------
\documentclass[aspectratio=169,xcolor=dvipsnames]{beamer}
\usetheme{Simple}
\usefonttheme{professionalfonts}

\usepackage{dsfont}
\usepackage{xfrac}

\usepackage{hyperref}
\usepackage{algorithm}
\usepackage{algpseudocode}

\usepackage{graphicx} % Allows including images
\usepackage{booktabs} % Allows the use of \toprule, \midrule and \bottomrule in tables

%----------------------------------------------------------------------------------------
%	TITLE PAGE
%----------------------------------------------------------------------------------------

% The title
\title[short title]{Magneto-optical Kerr Effect}

\author{Rohit Sharma \& Umur Can Kaya}
\institute[PTB] % Your institution may be shorthand to save space
{
    % Your institution for the title page
    Freie Universit\"{a}t Berlin\\
    Advanced Laboratory Course WS 21-22
    \vskip 3pt
}
\date{\today} % Date, can be changed to a custom date


%----------------------------------------------------------------------------------------
%	PRESENTATION SLIDES
%----------------------------------------------------------------------------------------

\begin{document}

\begin{frame}
    \titlepage
\end{frame}

%------------------------------------------------
\section{Problem statement}
%------------------------------------------------

\begin{frame}{Problem statement: Bayesian inference}

\end{frame}

\begin{frame}{Analytical solution}

\end{frame}

%------------------------------------------------
\section{MCMC algorithms of comparison}
%------------------------------------------------
\begin{frame}{Metropolis-Hastings (MH)}

\end{frame}

\begin{frame}{Preconditioned Crank-Nicolson (pCN)}

\end{frame}

\begin{frame}{Generalized preconditioned Crank-Nicolson (gpCN)}

\end{frame}

\begin{frame}{No-U-Turn sampler (NUTS)}

\end{frame}

\begin{frame}{MCMC algorithms}

\end{frame}

%------------------------------------------------
\section{Numerical illustration}
%------------------------------------------------

\begin{frame}{Numerical illustration: Bayesian linear regression}
 
\end{frame}

\begin{frame}{Data, model, and simulation parameters}


\end{frame}

%------------------------------------------------
\section{Results}
%------------------------------------------------

\begin{frame}{$C_0$: constant case}

\end{frame}

%------------------------------------------------
\section{Conclusions}
%------------------------------------------------

\begin{frame}{Conclusions}

\end{frame}

%------------------------------------------------
\section{References}
%------------------------------------------------

\begin{frame}{References}
\begin{enumerate}
    \item \textbf{gpCN paper:} Daniel Rudolf, Björn Sprungk, 2016, \textit{On a generalization of the preconditioned Crank-Nicolson Metropolis algorithm}, arXiv:1504.03461
    \item \textbf{NUTS paper:} Matthew D. Hoffman, Andrew Gelman, 2011, \textit{The No-U-Turn Sampler: Adaptively Setting Path Lengths in Hamiltonian Monte Carlo}, arXiv:1111.4246
    \item \textbf{NumPyro paper:} Du Phan, Neeraj Pradhan, Martin Jankowiak, 2019, \textit{Composable Effects for Flexible and Accelerated Probabilistic Programming in NumPyro}, arXiv:1912.11554
    \item \textbf{ESS calculation:} Geyer, Charles J. 1992. “Practical Markov Chain Monte Carlo.” Statistical Science, 473–83
\end{enumerate}

\end{frame}

%----------------------------------------------------------------------------------------

\end{document}